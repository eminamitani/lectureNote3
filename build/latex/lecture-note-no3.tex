%% Generated by Sphinx.
\def\sphinxdocclass{jsbook}
\documentclass[letterpaper,10pt,dvipdfmx]{sphinxmanual}
\ifdefined\pdfpxdimen
   \let\sphinxpxdimen\pdfpxdimen\else\newdimen\sphinxpxdimen
\fi \sphinxpxdimen=49336sp\relax

\usepackage[margin=1in,marginparwidth=0.5in]{geometry}


\usepackage{cmap}
\usepackage[T1]{fontenc}
\usepackage{amsmath,amssymb,amstext}

\usepackage{times}

\usepackage{longtable}
\usepackage{sphinx}

\usepackage{multirow}
\usepackage{eqparbox}

% Include hyperref last.
\usepackage{hyperref}
% Fix anchor placement for figures with captions.
\usepackage{hypcap}% it must be loaded after hyperref.
% Set up styles of URL: it should be placed after hyperref.
\urlstyle{same}
\renewcommand{\contentsname}{Contents:}

\renewcommand{\figurename}{図 }
\renewcommand{\tablename}{TABLE }
\renewcommand{\literalblockname}{LIST }

\def\pageautorefname{ページ}

\setcounter{tocdepth}{1}



\title{lecture-note-no3 Documentation}
\date{12月 11, 2018}
\release{1.0}
\author{Emi Minamitani}
\newcommand{\sphinxlogo}{}
\renewcommand{\releasename}{リリース}
\makeindex

\begin{document}

\maketitle
\sphinxtableofcontents
\phantomsection\label{\detokenize{index::doc}}



\chapter{多粒子系のハミルトニアンの実際と応用例}
\label{\detokenize{index:id1}}\label{\detokenize{index:id2}}
前回までの講義で、場の演算子で表記することで、多体量子系でのハミルトニアンが1粒子系と
似たような形で手に入ることや、場の演算子を生成消滅演算子と基底関数によって展開することで
生成消滅演算子で表記されたハミルトニアンが得られることを紹介した。
その具体的な応用例を紹介し、多体系での量子効果を扱う方法の感覚を紹介したい。


\section{電子電子相互作用}
\label{\detokenize{index:id3}}

\subsection{アンダーソンモデル}
\label{\detokenize{index:id4}}
電子電子相互作用を含むシンプルなモデルとしては、互いに相互作用をしない自由電子的な
伝導電子と、局在した軌道に存在するため強い電子-電子相互作用を感じる局在電子とが
相互作用をする場合を考えた、アンダーソンモデルがあげられる。
\begin{equation*}
\begin{split}H=\sum_{k\sigma} \epsilon_k c_{k\sigma}^\dagger c_{k\sigma}+\sum_{\sigma}E_d d_{\sigma}^\dagger
d_{\sigma} +Un_{d\uparrow}n_{d\downarrow} +\sum_{k\sigma}(V_{0}
c_{k\sigma}^\dagger d_{\sigma} + h.c.)\end{split}
\end{equation*}
\(\sum_{k\sigma} \epsilon_k c_{k\sigma}^\dagger c_{k\sigma}\)
は自由な伝導電子の部分を表している。
この項だけならば、それぞれの波数の電子が
\(\epsilon_k\)
のエネルギーを持つという対角的なハミルトニアンになっている。
しかし、その自由な伝導電子はトンネル効果によって局在軌道に入り込むことがある。
それを記述しているのが
\(\sum_{k\sigma}(V_{0} c_{k\sigma}^\dagger d_{\sigma} + h.c.)\)
の部分である。
\(Un_{d\uparrow}n_{d\downarrow}\)
は局在電子が感じる電子間相互作用である。
\(n_{d\uparrow}=d^\dagger_\uparrow d_\uparrow\)
のように電子数の演算子を書き直してやれば、前回の講義資料の最後に紹介した
2体のハミルトニアンに相当していることがわかるだろう。

このアンダーソンモデルは当初、非磁性金属中の磁性原子不純物が磁性を持つか否かを議論するためのモデルとして提案された。
提案者のアンダーソン自身は、
\(Un_{d\uparrow}n_{d\downarrow}\)
の2つの電子数演算子のうち、片方を平均値にしてしまうという平均場近似で、あるUの値を境に磁性を持つ場合と持たない場合に
区別されるということを提唱した。
しかし、このような不連続性は近似が単純であったために生じており、実際はより複雑な状況が生じている。
複雑さの原因は、十分に低温で熱ゆらぎが少ない場合には、
伝導電子と局在電子がトンネル効果によってすり替わる過程を介して
全体のエネルギーを下げることが許容される、伝導電子と局在電子がスピン1重項をとった状態が
現れてくることにある。伝導電子が1つ2つならこの1重項形成は大して複雑な問題ではないが、
実際に出てくるのは無数の伝導電子がゆるく局在電子と1重項を組んだ状態である。
この状態では伝導電子がゆるく局在電子に束縛されることになるので、
その結果、電気抵抗が上昇する。

通常の金属では電気抵抗は
温度を下げて行くと単調に減少し、
不純物散乱に由来する残留抵抗が最後に残る。
しかし、磁性不純物を含んだ金属では上記のように
冷やしていくと、一旦さがったのちに電気抵抗が上がるという、電気抵抗に極小点が現れる
という現象が起きる。これを近藤効果という。

近藤効果自体は、1920年ごろから発見されており、上に述べたメカニズムは
日本人の近藤淳先生や芳田圭先生らによって60年代にあらかた解明されていた。
しかし、量子多体状態を扱う困難により、厳密解や数値的に高精度な解が得られるようになったのは
80年代以降である。

最近でも、量子ドットの電子と、量子ドットにつながる電極の電子の間で
近藤効果が起きることや、分子系でも起きることなど、もともとの磁性不純物が入った金属以外に
様々な系で起きることが報告されており、今だに研究が続けられている問題の一つである。


\subsection{ハバードモデル}
\label{\detokenize{index:id5}}
アンダーソンモデルと似ているが、ありとあらゆるところで電子-電子相互作用が強力な場合に
相当するのがハバードモデルである。
ハバードモデルの場合は、局在電子と局在電子の間をホッピングする項と
局在電子が感じる電子-電子相互作用の2つの項からなる。
\begin{equation*}
\begin{split}H=\sum_{i j\neq i\sigma}( t_{ij\sigma} d_{i\sigma}^\dagger d_{j\sigma}+h.c.) +\sum_{i\sigma}Un_{d_i\uparrow}n_{d_i\downarrow}\end{split}
\end{equation*}
これはちょうど強束縛模型に
Indices and tables
==================
\begin{itemize}
\item {} 
\DUrole{xref,std,std-ref}{genindex}

\item {} 
\DUrole{xref,std,std-ref}{search}

\end{itemize}



\renewcommand{\indexname}{索引}
\printindex
\end{document}